\documentclass{ximera}
\usepackage{PackageLoader}
%\usepackage{sagetex}


%Renew commands

\renewcommand{\log}{\ln}

%% New commands


%%%%%%%%%%%%%%%%%%%%
% New Conditionals %
%%%%%%%%%%%%%%%%%%%%




\title{Practice Problems for Topic 9}
\begin{document}
\begin{abstract}
Topic 9 Unlimited Practice
\end{abstract}
\maketitle

%%% Topic: Mathematical Reasoning. Subtopics:
%   Clarifying questions
%   Extraneous vs vital information/data
%   Information vs Data


%%%%%%%%%%%%%%%%%%%%%%%%%%%%%%%%%%%%%%%%%%%%%%%%%%%%%%%%%%%%%%%%%%%%%%%%%%%%%%%%%
%%%%%%%%%                        Question Code                           %%%%%%%%
%%%%%%%%%%%%%%%%%%%%%%%%%%%%%%%%%%%%%%%%%%%%%%%%%%%%%%%%%%%%%%%%%%%%%%%%%%%%%%%%%


\begin{sagesilent}

#####Define default Sage variables.
#Default function variables
var('x,y,z,X,Y,Z')
#Default function names
var('f,g,h,dx,dy,dz,dh,df')
#Default Wild cards
w0 = SR.wild(0)


def RandInt(a,b):
    """ Returns a random integer in [`a`,`b`]. Note that `a` and `b` should be integers themselves to avoid unexpected behavior.
    """
    return QQ(randint(int(a),int(b)))
    # return choice(range(a,b+1))

def NonZeroInt(b,c, avoid = [0]):
    """ Returns a random integer in [`b`,`c`] which is not in `av`. 
        If `av` is not specified, defaults to a non-zero integer.
    """
    while True:
        a = RandInt(b,c)
        if a not in avoid:
            return a



# \end{sagesilent}
\begin{sagesilent}
q1c1 = RandInt(-5,5)
q1c2 = RandInt(-5,5)
q1c3 = RandInt(-5,5)
q1c4 = RandInt(-5,5)
q1c5 = RandInt(0,1)
q1c6 = RandInt(0,1)
q1c7 = RandInt(1-q1c6,1)
q1c8 = RandInt(1-q1c5,1)

q1f1 = q1c5*(x - q1c1 - 1) + 1
q1f2 = q1c6*(x - q1c2 - 1) + 1
q1f3 = q1c7*(x - q1c3 - 1) + 1
q1f4 = q1c8*(x - q1c4 - 1) + 1

q1p = expand(q1f1*q1f2*q1f3*q1f4)

q1ans1 = q1c5 + q1c6 + q1c7 + q1c8
q1ans2 = q1c5*q1c1 + q1c6*q1c2 + q1c7*q1c3 + q1c8*q1c4

\end{sagesilent}

\begin{problem}
Consider the following polynomial:
\[
p(x) = \sage{q1p(x)}
\]

Fully factor this polynomial to answer the following questions;

How many roots does this polynomial have (counting multiplicity; ie if it has a factor of $x^2$, then that counts as 2 roots for this question). \\
$\answer{\sage{q1ans1}}$

What is the sum of the zeros of the polynomial?\\
$\answer{\sage{q1ans2}}$

What is the fully factored form of the polynomial?\\
$\answer{\sage{(q1f1*q1f2*q1f3*q1f4)(x)}}$

\end{problem}



\begin{sagesilent}
q2c1 = RandInt(1,5)

q2f1 = x^4 - q2c1^4

q2ans1 = 2*(q2c1)^(1/2)
q2ans2 = x^2 + q2c1^2
q2ans3 = x + q2c1
q2ans4 = x - q2c1

\end{sagesilent}

\begin{problem}
Consider the following polynomial:
\[
p(x) = \sage{q2f1(x)}
\]

Fully factor this polynomial to answer the following questions;

How many real roots does this polynomial have (counting multiplicity; ie if it has a factor of $x^2$, then that counts as 2 roots for this question). \\
$\answer{2}$

\begin{problem}
    What are the (real) zeros of the polynomial? (Order from smallest to largest)\\
    $\answer{\sage{-q2c1}}, \answer{\sage{q2c1}}$
\end{problem}

How many non-real zeros of the polynomial are there?\\
$\answer{2}$

\begin{problem}
    What are the (non-real) zeros of the polynomial? (Order from smallest coefficient of $i$ to the largest)\\
    $\answer{\sage{-q2c1}}i, \answer{\sage{q2c1}}i$
\end{problem}

How many terms (with real-valued coefficients) does $p(x)$ have when it is factored (without using complex values)?\\
$\answer{3}$
\begin{problem}

What is the factored form of the polynomial using only real-valued coefficients?\\
Enter each term ordering highest degree to lowest degree terms, and ordering the single degree roots from left to right starting with the smallest and going up to largest zeros. For example: If you factor your polynomial to $p(x) = (x^2 + 1)(x+17)(x-13)(x-1)$ then you would enter them in the order: $(x^2+1)(x+17)(x-1)(x-13)$.\\
$(\answer{\sage{q2ans2}})(\answer{\sage{q2ans3}})(\answer{\sage{q2ans4}})$
\end{problem}
\end{problem}





\begin{sagesilent}
q3c1 = NonZeroInt(-5,5)

q3f1 = x^3 - q3c1^3

q3ans1 = -q3c1 / 2 - 3^(1/2)*(q3c1 / 2)*i
q3ans2 = -q3c1 / 2 + 3^(1/2)*(q3c1 / 2)*i

q3ans3 = x^2 + q3c1*x + q3c1^2
q3ans4 = x - q3c1

\end{sagesilent}

\begin{problem}
Consider the following polynomial:
\[
p(x) = \sage{q3f1(x)}
\]

Fully factor this polynomial to answer the following questions;

How many real roots does this polynomial have (counting multiplicity; ie if it has a factor of $x^2$, then that counts as 2 roots for this question). \\
$\answer{1}$

\begin{problem}
    What are the (real) zeros of the polynomial? (Order from smallest to largest)\\
    $\answer{\sage{q3c1}}$
\end{problem}

How many non-real zeros of the polynomial are there?\\
$\answer{2}$

\begin{problem}
    What are the (non-real) zeros of the polynomial? (Order from smallest coefficient of $i$ to the largest)\\
    $\answer{\sage{q3ans1}}, \answer{\sage{q3ans2}}$
\end{problem}

How many terms (with real-valued coefficients) does $p(x)$ have when it is factored (without using complex values)?\\
$\answer{2}$
\begin{problem}

What is the factored form of the polynomial using only real-valued coefficients?\\
Enter each term ordering highest degree to lowest degree terms, and ordering the single degree roots from left to right starting with the smallest and going up to largest zeros. For example: If you factor your polynomial to $p(x) = (x^2 + 1)(x+17)(x-13)(x-1)$ then you would enter them in the order: $(x^2+1)(x+17)(x-1)(x-13)$.\\
$(\answer{\sage{q3ans3}})(\answer{\sage{q3ans4}})$
\end{problem}
\end{problem}


\begin{sagesilent}
q4c1 = RandInt(-5,5)
q4c2 = RandInt(-5,5)
q4c3 = RandInt(-5,5)
q4c4 = RandInt(-5,5)
q4c5 = NonZeroInt(-3,3)
q4c6 = NonZeroInt(-3,3)
q4c7 = NonZeroInt(-3,3)
q4c8 = NonZeroInt(-3,3)

q4f1 = q4c5*x - q4c1
q4f2 = q4c6*x - q4c2
q4f3 = q4c7*x - q4c3
q4f4 = q4c8*x - q4c4

q4p = expand(q4f1*q4f2*q4f3*q4f4)

q4ans1 = q4c1/q4c5 + q4c2/q4c6 + q4c3/q4c7 + q4c4/q4c8
q4ans2(x) = (q4f1*q4f2*q4f3*q4f4)(x)
\end{sagesilent}

\begin{problem}
Consider the following polynomial:
\[
p(x) = \sage{q4p(x)}
\]

Fully factor this polynomial to answer the following questions;

How many roots does this polynomial have (counting multiplicity; ie if it has a factor of $x^2$, then that counts as 2 roots for this question). \\
$\answer{4}$

What is the sum of the zeros of the polynomial?\\
$\answer{\sage{q4ans1}}$

What is the fully factored form of the polynomial?\\
$\answer{\sage{q4ans2(x)}}$

\end{problem}











\end{document}